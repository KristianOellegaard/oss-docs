\input{mmd-article-header}
\def\mytitle{BOSH Documentation}
\def\myauthor{VMware 2012 - Cloud Foundry}
\def\latexmode{memoir}
\input{mmd-article-begin-doc}
\chapter{Introduction to BOSH}
\label{introductiontobosh}

BOSH is a framwork and tool-chain for release engineering, deployment
and life cycle management of distributed services. In this manual we
describe the architecture, topology, configuration and use of BOSH, as
well as the structure and conventions used in packaging and deployment.

BOSH introduces a fairly prescriptive way of managing systems and
services. It was originally developed in the context of the Cloud
Foundry Application Platform as a Service, but even if this has been the
primary consumer, the framework is general purpose and can be used to
deploy many different

TODO: what is BOSH not

\chapter{BOSH Component Definitons}
\label{boshcomponentdefinitons}

\section{Infrastructure as a Service (IaaS)}
\label{infrastructureasaserviceiaas}

The core BOSH engine is abstracted away from any particular
Infrastructure as a Service (IaaS), such as VMware vSphere, AWS or
OpenStack. The interface to thse is implemented as plugins to BOSH. At
the time of writing only the vSphere implementation is fully featured.

\section{Cloud Provider Interface (CPI)}
\label{cloudproviderinterfacecpi}

As a user of BOSH you're not directly exposed to the the BOSH Cloud
Provider Interface, but it can be helpful to understand it's primitives
when learning how BOSH works.

create\_stemcell
delete\_stemcell
create\_vm
delete\_vm
configure\_networks
create\_disk
delete\_disk
attach\_disk
detach\_disk

Please refer to the API documentation +director\slash lib\slash director\slash cloud.rb+
for a detailed explanation of the CPI primitives.

The CPI is used primarily to do low level creation and management of
resources in an IaaS, once a resouce is up and running command and
control is handed over to the higher level BOSH Director-Agent
interaction.

\section{BOSH Director}
\label{boshdirector}

The Director is the core orchestrating component in BOSH which controls
creation of VMs, deployment and other life cycle events of software and
services.

\section{BOSH CLI}
\label{boshcli}

TODO

\section{BOSH Agent}
\label{boshagent}

Every VM created by BOSH includes a an Agent, which does initial
configuration of a system once the Director has created it through the
CPI and further install software and services, once the Director
instructs it to apply a Job.

\section{Stemcells}
\label{stemcells}

A BOSH stemcell is typically a simple VM template with an embedded BOSH
Agent. These are uploaded using the BOSH CLI and used by the Director
when creating VMs through the CPI. When the Director create a VM through
the CPI, it will pass along configurations for networking and storage as
well as the location and credentials for the BOSH Message Bus and the
BOSH Blobstore.

\section{Releases}
\label{releases}

A Release in BOSH is a packaged bundle of service descriptors (known as
Jobs in BOSH), a collection of software bits and configurations. A
release contains all the static bits (source or binary) required to have
BOSH manage an application or a distributed service. A Release is
typcially not restricted to any particular environment an as such it can
be re-used across clusters handling different stages in a service life
cycle, such as development, QA, staging or production. The BOSH CLI
manages both the creation of releases and the deployments into specific
environment.

\section{Deployments}
\label{deployments}

While BOSH Stemcells and Releases are static compnents, we say that they
are bound together into a Deployment by what we call a Deployment
Manifest. In the Deployment Manifest you declare pools of VMs, which
networks they live on, which Jobs (service componens) from the Release
you want to activate. Job configuration specify life cycle parameters,
the number instances of a Job, as well as network and storage
requirements. In the Deployment Manifest you can also speficy properties
at various levels used to paramaterize configuration templates contained
in the Release.

Using the BOSH CLI you specify a Deployment Manifest and perform a
Deploy operation (+bosh deploy+), which will take this specification and
go out to your cluster and either create or update resources in
accordance to the specification.

\section{Blobstore}
\label{blobstore}

\section{BOSH Monitor}
\label{boshmonitor}

\section{Message bus}
\label{messagebus}

\chapter{Installing BOSH}
\label{installingbosh}

TODO: replace this with gem install of cli gem
TODO: remove this section when we don't need chef\_deployer anymore

BOSH is a Ruby based toolchain and we suggest that you are set up with
the following

\begin{itemize}
\item rbenv available at \href{https://github.com/sstephenson/rbenv}{https:/\slash github.com\slash sstephenson\slash rbenv}\footnote{\href{https://github.com/sstephenson/rbenv}{https:/\slash github.com\slash sstephenson\slash rbenv}}

\item Ruby 1.9.2

\end{itemize}

\chapter{Configure BOSH Director}
\label{configureboshdirector}

[NOTE]
The current +chef-solo+ based installer is being re-written as a
mini-bosh instance.

To install BOSH into an infrastructure we currently assume that the
target VMs have been created.

TODO: check if we can provide vm\_builder instructions for creating and
/\slash uploading these to IaaS.

\begin{adjustwidth}{2.5em}{2.5em}
\begin{verbatim}

    ~/projects/deployments/mycloud/cloud
      assets/
        director/
          director.yml.erb           <1>
          chef.rb                    <2>
          config.yml                 <3>

cd ~/projects/bosh/chef_deployer
rake install

cd ~/projects/bosh/release
chef_deployer deploy ~/projects/deployments/mycloud/cloud

\end{verbatim}
\end{adjustwidth}

\chapter{BOSH CLI}
\label{boshcli}

Go Oleg

\chapter{Some section that has stemcells + releases \slash  Director interaction}
\label{somesectionthathasstemcellsreleasesdirectorinteraction}

upload

\chapter{Releases}
\label{releases}

\section{Release Repository}
\label{releaserepository}

A BOSH release is built from a directory tree following a structure
described in this section:

\section{Jobs}
\label{jobs}

TODO: job templates
TODO: use of properties
TODO: ``the job of a vm''
TODO: monitrc (gonit)
TODO: DNS support

\section{Packages}
\label{packages}

TODO: ishisness!
TODO: compilation
TOOD: dependencies
TODO: package specs

\section{Sources}
\label{sources}

final release

\section{Blobs}
\label{blobs}

TODO: configuration options for Blobstore (Atmos vs S3)

\section{Versioning schemes}
\label{versioningschemes}

\section{Configuring Releases}
\label{configuringreleases}

\section{Building Releases}
\label{buildingreleases}

\section{Final Releases}
\label{finalreleases}

\chapter{BOSH Deployments}
\label{boshdeployments}

TODO: capture all the steps that the deployment does

\section{BOSH Property Store}
\label{boshpropertystore}

\section{BOSH Deployment Manifest}
\label{boshdeploymentmanifest}

TODO: options global\slash job propertes
TODO: cloud\_properties for the cli

\chapter{BOSH Troubleshooting}
\label{boshtroubleshooting}

TODO: cloud check
TODO: BOSH SSH
TODO: logs

\input{mmd-memoir-footer}

\end{document}
